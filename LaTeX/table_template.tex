\documentclass{article}
\usepackage{csvsimple,booktabs,array,siunitx,tabu}
\usepackage{filecontents}

% Enable siunitx S-columns to be used with tabu X column.
\ExplSyntaxOn
\cs_new_eq:NN \siunitx_table_collect_begin:Nn \__siunitx_table_collect_begin:Nn
\ExplSyntaxOff
\newcommand{\ra}[1]{\renewcommand{\arraystretch}{#1}}

\sisetup{
    group-four-digits=true,
    group-separator={,},
    round-precision=2,
    round-mode=places
}

\begin{filecontents*}{grades.csv}
height,name,mat,gender,grade
654e3,Maier,12345,m,2.6e3
99.6,Huber,23456,f,4.234e2
754.0,Weisbaeck,34567,m,7.5e7
\end{filecontents*}

\begin{document}


% Usage
% 	CSVSimple does not accept S columns as first column.
% 	If column 1 needs SIunitx alignment then use the \tablenum macro.

\begin{table*}[htbp] 
\centering
\ra{1.1}
\begin{tabu} to \linewidth {@{}
    X[c]
    X[c]
    X[c]{S[table-format=5]}
    X[c]
    X[c]{S[table-format=1.2e1]}
    @{}}
    \toprule
    & \multicolumn{2}{c}{\textbf{HB2 WT}} & \multicolumn{2}{c}{\textbf{MCF7}} \\
    \cmidrule{2-3} \cmidrule{4-5}
    & {1} & {2} & {1} & {2} \\
    \midrule
    \csvreader[late after line=\\]{grades.csv}{}
        {\tablenum[table-format=3.2e1]{\csvcoli} & 
        \csvcolii & \csvcoliii & 
        \csvcoliv & \csvcolv}
    \bottomrule
\end{tabu}
\end{table*}

\begin{table*}[htbp] 
\centering
\begin{tabu} to \linewidth {@{}XXXXX@{}}
    \toprule
    height & name & mat & gender & grade \\
    \midrule
    \csvreader[late after line=\\]{grades.csv}{}
        {\csvcoli & \csvcolii & \csvcoliii & \csvcoliv & \csvcolv}
    \bottomrule
\end{tabu}
\end{table*}

\end{document}